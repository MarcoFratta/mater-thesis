%----------------------------------------------------------------------------------------
\chapter{Evaluation}
\label{chap:evaluation}
%----------------------------------------------------------------------------------------

The evaluation of the Alchemist DSL focuses on verifying that the solution meets 
the functional and non-functional requirements outlined in \Cref{chap:analysis}. 
The assessment is divided into two main parts: 
a technical validation through two different testing strategies,
 and a qualitative discussion on the benefits introduced by the new system.

\section{Testing Strategy}
\label{sec:testing}

The testing process was designed to ensure correctness, equivalence with the legacy system, and performance stability. 
The strategy followed an incremental approach, starting from the unit testing of individual DSL components,
 proceeding to the structural validation of the generated simulation models, 
 and concluding with runtime behavior comparison and performance testing.

\subsection{Unit Testing}
\label{subsec:unit-testing}

The unit testing phase focused on validating the core architectural mechanisms of the DSL rather than exhaustively 
covering every possible configuration option. 
Given that the DSL relies on a consistent recursive builder pattern, the strategy prioritized the verification of 
the most complex and critical infrastructure components: the variable management system and the 
type-safe context propagation.

The \textbf{Variables Context} received the most extensive unit testing coverage due to its central 
role in enabling batch executions. 
Unit tests specifically targeted this component to ensure that the delegation mechanism correctly captures default 
values during the definition phase and that the \texttt{DSLLoader} successfully injects run-specific values
 during the materialization phase. 
This isolation ensured that the complex logic handling dependent variables and potential circular dependencies was
 robust before being integrated into the full simulation loader.

Beyond variables, unit tests were employed to validate the fundamental building blocks of the DSL structure,
 such as the root \texttt{SimulationContext} and the primary nested contexts like \texttt{Deployments Context}.
The primary objective was to verify the correct propagation of the generic types $T$ (concentration) and $P$ (position)
 throughout the context hierarchy. 
These tests confirmed that the Kotlin compiler and the DSL infrastructure correctly enforced type constraints,
 preventing type mismatches within nested blocks.

However, an exhaustive unit test for every specific domain entity (e.g., every available reaction, condition, or deployment type)
 was deemed redundant at this stage. 
Since the DSL leverages a uniform builder pattern, once the core mechanisms, such as context nesting, property delegation, 
and builder execution, were validated on the main components, the correctness of the broader domain coverage was
 delegated to the \textbf{Equivalence Testing} phase. 
This strategy avoided determining the correctness of individual wrappers, 
relying instead on the comprehensive verification against the legacy YAML system to catch any discrepancies in the generated models.

\subsubsection{Processor Verification}

In addition, the Kotlin Symbol Processing (KSP) component (described in \Cref{sec:ksp-implementation}) 
required a dedicated testing strategy. 
Since the processor is responsible for generating the bridge code that makes the DSL more user-friendly,
 its correctness is a key factor for the user experience.

The unit testing for the processor involved the compilation of synthetic Kotlin sources annotated with \texttt{@AlchemistKotlinDSL}. 
The tests asserted that the processor:
\begin{itemize}
    \item Correctly identified the primary constructors of annotated classes.
    \item Accurately distinguished between user-configurable parameters and system-injectable
     dependencies (e.g., \texttt{Environment}, \texttt{RandomGenerator}, etc.).
    \item Generated factory functions with the correct signature, including appropriate context receivers and generic constraints.
    \item Correctly synthesized the accessor paths (e.g., mapping a request for `Environment` to the correct parent context property)
     to solve the context chaining problem.
\end{itemize}

\subsection{Equivalence Testing}
\label{subsec:equivalence-testing}

To demonstrate compliance with the \textbf{YAML equivalence} requirement, an integration test suite was developed. 
The objective was to prove that for any valid YAML configuration, a semantically \textbf{identical} DSL script exists and produces 
an indistinguishable simulation model.
This verification relies on a set of \textit{reference scenarios}, covering the full spectrum of Alchemist features, 
from simple node placement to complex environment-aware movement and reaction programs.
This type of test is the core tool to verify the correctness of the DSL. It \textbf{must} be used to 
ensure that the DSL is able to produce the same simulation model as the YAML loader. 
The strategy used to test the DSL was to adopt a bottom-up approach, starting from the simplest scenarios 
and gradually adding complexity. This approach allowed identifying and fixing issues early on, 
giving a clear path to the final solution.

The validation process utilizes a custom comparison framework, which performs verification at two levels:
\begin{itemize}
    \item \textbf{Static Structural Comparison}: verifies that the DSL produces a simulation model 
    identical to the YAML loader's output before execution begins.
    \item \textbf{Runtime Behavioral Comparison}: run the simulation for a certain number of steps or until a certain time,
    and then compare the final state of the simulation.
\end{itemize}

\subsubsection{Static Structural Comparison}

The \texttt{StaticComparisonHelper} verifies that the DSL produces a simulation 
model identical to the YAML loader's output before execution begins. 
It inspects the \texttt{Loader} objects produced by both systems and compares the resulting simulation instances deeply.
Key verification points include:
\begin{itemize}
    \item \textbf{Variable Metadata}: asserting that the set of declared variables and their default values are identical.
    \item \textbf{Environment Configuration}: checking that node counts and positions match.
    \item \textbf{Node State}: verifying that every node is initialized at the exact same
    coordinates and contains the same set of molecules and concentrations.
    \item \textbf{Program Structure}: inspecting the reactions attached to nodes to ensure they have the same conditions,
     actions, and time distributions.
\end{itemize}
This phase ensures that the DSL syntax correctly maps to the underlying domain model structure.
However, this static analysis approach encountered significant limitations due to the legacy nature of the Alchemist codebase. 
Many core interfaces and classes do not expose their full internal state or do not implement value-based \texttt{equals()} methods, 
relying instead on reference equality. 
Verifying that two distinct instances of a complex object (e.g., a \texttt{TimeDistribution} or a \texttt{Layer}) are semantically 
identical proved difficult without resorting to fragile reflection-based inspection of private fields. 

To mitigate this, the static comparison often relies on proxy indicators, such as verifying that the class types 
match (via class name inspection) or comparing the string representation (\texttt{toString()}) of components. 
While useful for catching gross configuration errors, 
these checks cannot guarantee that two objects have identical internal configurations if their string representations are incomplete.

\subsubsection{Runtime Behavioral Comparison}

To overcome the limitations of static inspection and guarantee true semantic equivalence,
the evaluation strategy introduced a \textbf{Runtime Behavioral Comparison}. 
The rationale is that if two simulations are configured identically, 
and the underlying simulator is deterministic (given the same random seed), 
they must evolve through identical states over time.

The \texttt{RuntimeComparisonHelper} executes both the YAML-generated and DSL-generated simulations,
 stepping them forward for a fixed duration or a specific number of steps. 
At the end of this execution, the framework compares the tangible, observable state of the simulation—properties that are 
publicly accessible and critical to the simulation's outcome:
\begin{itemize}
    \item \textbf{Global State}: Asserting that both simulations reached the exact same simulation time and step count.
    \item \textbf{Node Positions}: Verifying that the coordinates of every node match, 
    confirming that movement logic and environment boundaries are the same.
    \item \textbf{Node Contents}: Checking that the set of molecules and their concentrations inside each node are identical,
     confirming that reactions and diffusion processes executed in the same order and with the same effects.
\end{itemize}
This dynamic verification provides a much stronger guarantee of correctness than static analysis. 
It confirms not only that the components are of the right type, but that they were initialized with
 the correct parameters and that the Random Number Generators (RNGs) were consumed in the exact same 
 sequence, a crucial factor for reproducibility in stochastic simulations.

\subsection{Performance Testing}
\label{subsec:performance-testing}

While the primary goal of the DSL is usability and safety, the \textbf{performance} of the loading phase must 
remain within acceptable limits. 
The \texttt{PerformanceComparisonTest} suite benchmarks the initialization time of the DSL against the YAML loader.
The tests measure the time required to parse the configuration and produce a ready-to-run \texttt{Simulation} object.
While the YAML loader is slightly faster than the DSL in raw loading speed, due to the overhead of Kotlin script compilation
and class loading, the tests aim to verify that the overhead remains acceptable.
The results show that while there is a distinct initial \textit{warm-up} cost for the Kotlin script engine,
the impact on the overall workflow is mitigated by the batch execution model, 
where the script is compiled only once for an arbitrary number of simulation runs.
The tests verify that the overhead remains acceptable and that the DSL is not significantly slower than the YAML loader.
The test was made using a moderately complex simulation configuration, as shown in \Cref{lst:complextestsim}.



\begin{center}
    \begin{minipage}{0.48\textwidth}
    \inputminted[
      fontsize=\tiny,
      breaklines,
      firstline=1,
      lastline=42, 
      linenos
    ]{kotlin}{listings/complextestsim.kts}
    \end{minipage}
    \hfill
    \begin{minipage}{0.48\textwidth}
    \inputminted[
      fontsize=\tiny,
      breaklines,
      firstline=43,
      lastline=80, 
      linenos
    ]{kotlin}{listings/complextestsim.kts}
    \end{minipage}
    \captionof{lstlisting}
    {Simulation configuration used for the performance test}
    \label{lst:complextestsim}
\end{center}
\newpage
    



\Cref{tab:performance-results} presents the performance comparison results obtained 
from 20 iterations of both the loading phase, and the action phase separately.
The test measures the time required to load and initialize the simulation 
configuration using both the YAML loader and the DSL loader.


\begin{table}[h]
\centering
\caption{Performance comparison results (20 iterations)}
\label{tab:performance-results}
\begin{tabular}{l|r|r}
\textbf{Phase} & \textbf{YAML (ms)} & \textbf{DSL (ms)} \\
\hline
\textbf{Loading Phase} & 11,002 & 15,802 \\
\hline
\textbf{Action Phase} & 4,422 & 2,734 \\
\hline
\textbf{Total Average} & \textbf{15,424} & \textbf{18,536} \\
\end{tabular}
\end{table}

The results reveal an interesting performance profile: while the DSL is slower during the \textbf{Loading Phase}, 
it performs better than the YAML loader during the \textbf{Action Phase}.
This performance difference can be attributed to the fact that the Loading Phase includes the overhead of Kotlin script compilation, 
which is a one-time cost that occurs when the script is first processed.
However, once the configuration is loaded, the DSL's materialization process is more efficient, resulting in faster Action Phase execution.
The overall performance shows that YAML is approximately 20\% faster in total time, but this overhead is acceptable given the benefits in terms of type safety, 
IDE support, and code reusability.
Moreover, in batch execution scenarios where the script is compiled once and used for multiple simulation runs,
 the Loading Phase cost is amortized, making the DSL's superior Action Phase performance more significant.

\section{Experimental Evaluation}
\label{sec:experimental-evaluation}

Beyond technical correctness, the success of the DSL is measured by how well it fulfills the user-centric 
goals of usability, 
type safety, and expressiveness. 
This section evaluates the impact of the new system on the developer experience, 
comparing it qualitatively against the legacy YAML-based workflow.

\subsection{DSL Usability and Expressiveness}
\label{subsec:usability}

The adoption of Kotlin as the host language transforms the configuration process from writing passive data 
descriptions (YAML) 
to developing active simulation software. 
This shift introduces several key improvements in usability:

\begin{itemize}
    \item \textbf{IDE-Assisted Development}: Unlike YAML, where users must rely on external documentation or 
    schema validators (often incomplete), the DSL benefits from the full power of the IDE. 
    Features such as \textbf{auto-completion} allow users to explore the available API surface interactively. 
    For instance, typing \texttt{this.} instantly suggests all available properties and methods of the current context,
    drastically reducing the need to switch context to read the documentation.
    
    \item \textbf{Type Safety as Feedback}: The static type system acts as an immediate feedback loop. 
    Errors such as assigning a \texttt{Double} to an \texttt{Int} parameter, or attempting to use a 2D position in
     a generic implementation, are flagged at compile-time (or edit-time in the IDE) rather 
     than crashing the simulation at runtime. Furthermore, users can leverage standard Kotlin language constructs,
     such as \texttt{for} loops, \texttt{if} statements, and other control flow primitives, enabling more expressive and flexible
     simulation configurations.

    \item \textbf{Scope Safety}: The use of \texttt{@AlchemistDsl} annotations (as detailed in \Cref{chap:implementation}) 
    successfully solved the structural ambiguity problem. 
    The compiler now enforces the domain hierarchy, making it syntactically impossible to nest 
    components incorrectly (e.g., defining a node inside a reaction). 
    This provides a structural guarantee that the YAML parser could only approximate through runtime validation.
\end{itemize}

One of the most powerful advantages of using Kotlin as the host language is the ability to create reusable 
components and functions that can be shared across multiple simulation files. 
Unlike YAML, where configuration must be duplicated or manually copied between files, 
the DSL enables users to define common patterns as functions or extension functions, 
which can be imported and reused throughout their project. 
This approach promotes code reuse, reduces duplication, and ensures consistency across simulations.

Consider, for example, a scenario where multiple simulations require a similar deployment pattern, 
such as a grid-based node placement with a specific reaction configuration. 
In YAML, this pattern would need to be replicated in each configuration file, 
making it difficult to maintain consistency and apply changes uniformly. 

\begin{minted}[fontsize=\scriptsize, linenos]{YAML}
incarnation: sapere
environment: { type: OSMEnvironment }
network-model: { type: ConnectWithinDistance, parameters: [1000] }
_venice_lagoon: &lagoon
  [[45.2038121, 12.2504425], [45.2207426, 12.2641754], [45.2381516, 12.2806549],
   [45.2570053, 12.2895813], [45.276336, 12.2957611], [45.3029049, 12.2991943],
   [45.3212544, 12.3046875], [45.331875, 12.3040009], [45.3453893, 12.3040009],
   [45.3502151, 12.3156738], [45.3622776, 12.3232269], [45.3719259, 12.3300934],
   [45.3830193, 12.3348999], [45.395557, 12.3445129], [45.3998964, 12.3300934],
   [45.4018249, 12.3136139], [45.4105023, 12.3122406], [45.4167685, 12.311554],
   [45.4278531, 12.3012543], [45.4408627, 12.2902679], [45.4355628, 12.2772217],
   [45.4206242, 12.2703552], [45.3994143, 12.2744751], [45.3738553, 12.2676086],
   [45.3579354, 12.2614288], [45.3429763, 12.2497559], [45.3198059, 12.2408295],
   [45.2975921, 12.2346497], [45.2802014, 12.2408295], [45.257972, 12.233963],
   [45.2038121, 12.2504425]]
deployments:
  type: Polygon
  parameters: [500, *lagoon]
  programs:
    - time-distribution: 10
      type: Event
\end{minted}
\captionof{lstlisting}{Example of a YAML simulation script with a complex deployment}

With the Kotlin DSL, users can define a reusable function that encapsulates this pattern:

\begin{minted}[fontsize=\scriptsize, linenos]{kotlin}
fun createLagoon() = listOf(
    45.2038121 to 12.2504425, 45.2207426 to 12.2641754, 45.2381516 to 12.2806549,
    45.2570053 to 12.2895813, 45.276336 to 12.2957611, 45.3029049 to 12.2991943,
    45.3212544 to 12.3046875, 45.331875 to 12.3040009, 45.3453893 to 12.3040009,
    45.3502151 to 12.3156738, 45.3622776 to 12.3232269, 45.3719259 to 12.3300934,
    45.3830193 to 12.3348999, 45.395557 to 12.3445129, 45.3998964 to 12.3300934,
    45.4018249 to 12.3136139, 45.4105023 to 12.3122406, 45.4167685 to 12.311554,
    45.4278531 to 12.3012543, 45.4408627 to 12.2902679, 45.4355628 to 12.2772217,
    45.4206242 to 12.2703552, 45.3994143 to 12.2744751, 45.3738553 to 12.2676086,
    45.3579354 to 12.2614288, 45.3429763 to 12.2497559, 45.3198059 to 12.2408295,
    45.2975921 to 12.2346497, 45.2802014 to 12.2408295, 45.257972 to 12.233963,
    45.2038121 to 12.2504425,
)
\end{minted}
\captionof{lstlisting}{Example of a reusable deployment function}

The \textit{createLagoon} function can now be used to create a lagoon deployment in any simulation script:

\begin{minted}[fontsize=\footnotesize, linenos]{kotlin}
simulation(incarnation) {
    networkModel = ConnectWithinDistance(1000.0)
    deployments {
        val lagoon = createLagoon()
        deploy(polygon(500, lagoon)) {
            programs {
                all {
                    timeDistribution("10")
                    reaction = event()
                }
            }
        }
    }
}
\end{minted}

This approach makes the configuration more readable and maintainable. 
Modifying the lagoon deployment requires only updating the \textit{createLagoon} function, 
without the need to update all configurations that use it.

Another approach is to directly define new DSL blocks to match the simulation domain. 
For example, a \textit{lagoon} block can be defined as follows:

\begin{minted}[fontsize=\footnotesize, linenos]{kotlin}
simulation(incarnation) {
    networkModel = ConnectWithinDistance(1000.0)
    deployments {
        lagoon(500) {
            programs {
                all {
                    timeDistribution("10")
                    reaction = event()
                }
            }
        }
    }
}
\end{minted}

The key benefit of this approach is that users can customize the DSL to match their specific simulation 
domain by creating reusable libraries or modules. 
These domain-specific extensions can define functions and DSL blocks with names and structures 
that align with the terminology and concepts of the target domain, 
such as \texttt{lagoon}, \texttt{swarm}, or \texttt{agent}. 
This domain-driven customization significantly enhances the readability of simulation scripts, 
as the syntax becomes more intuitive and self-documenting for domain experts. 
Furthermore, it provides users with greater control over the DSL syntax, 
enabling them to create abstractions that precisely match their modeling needs 
while maintaining full type safety and IDE support.

Finally, users can leverage standard Kotlin language features to create more complex simulations. 
For example, they can use \texttt{for} loops or \texttt{if} statements to create more complex situations 
without manually listing all possible cases.
\begin{minted}[fontsize=\footnotesize, linenos]{kotlin}
simulation(incarnation, { env }) {
    deployments {
        for (i in 1..10) {
            for (j in 1..10) {
                if (i == j) continue
                deploy(point(i.toDouble(), j.toDouble())){
                    // ...
                }
            }
        }
    }
}
\end{minted}

This example demonstrates only a simple usage of standard Kotlin language features to create more complex simulations. 
However, this design allows users to fully customize the DSL to match their specific simulation domain.

\subsection{Impact of the KSP Processor}
\label{subsec:ksp-evaluation}

The introduction of the KSP processor was a \textbf{decisive} factor in making the DSL usable in practice. 
Without the automatic generation of bridge code, the DSL would have suffered from severe verbosity, 
requiring users to manually inject system dependencies (like \texttt{environment} or \texttt{randomGenerator}) into every component constructor.
In the example below (\Cref{lst:example-verbose}), the usage of a \textit{BrownianMove} reaction would have been very verbose, since the reaction 
block is deeply nested in the context hierarchy. In this case, the reaction also requires several internal parameters to be passed to the constructor, 
reducing the readability of the code.


\begin{minted}[fontsize=\footnotesize, linenos]{kotlin}
simulation(incarnation) {
    networkModel = ConnectWithinDistance(1000.0)
    deployments {
        lagoon(500) {
            programs {
                all {
                    timeDistribution("10")
                    reaction = Event(node, timeDistribution)
                    addAction {
                        BrownianMove(
                            env,
                            node,
                            ctx.ctx.ctx.ctx.simulationGenerator,
                            0.0005,
                        )
                    }
                }
            }
        }
    }
}
\end{minted}
\label{lst:example-verbose}
\captionof{lstlisting}{Example of a simulation script with no helper functions}

With the KSP processor, it generates a helper function that can be used to create the reaction (as shown in \Cref{lst:example-helper}), 
simplifying the code and improving the readability. It also avoids the need to manually inject the system dependencies into the constructor, 
since they are injected automatically by the helper function.
\newpage
\begin{minted}[fontsize=\footnotesize, linenos]{kotlin}
simulation(incarnation) {
    networkModel = ConnectWithinDistance(1000.0)
    deployments {
        lagoon(500) {
            programs {
                all {
                    timeDistribution("10")
                    reaction = event()
                    addAction {
                        brownianMove(0.0005)
                    }
                }
            }
        }
    }
}
\end{minted}
\label{lst:example-helper}
\captionof{lstlisting}{Example of simulation script with helper functions}

Note that, by default, the generated functions are defined using the same name of the object they are generating, 
but with the first letter in lowercase. This should simplify the usage of the helper functions, since users can just
use the lowercase version of the object name to call the helper function. 
