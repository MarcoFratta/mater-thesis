%----------------------------------------------------------------------------------------
\chapter{Introduction}
\label{chap:introduction}
%----------------------------------------------------------------------------------------

The simulation of complex systems has become a fundamental tool in modern scientific research and engineering practice. 
Complex systems are characterized by the interaction of numerous components that give rise to emergent behaviors that cannot be easily predicted from the individual components alone. 
These systems span diverse domains, from biological networks and chemical reactions to distributed computing systems, social dynamics, and swarm robotics. 
Understanding and predicting the behavior of such systems requires sophisticated simulation frameworks capable of modeling spatial distributions, temporal dynamics, and stochastic interactions.

Alchemist is a simulation framework designed to address these challenges by providing a powerful platform for modeling spatially-distributed systems with temporal dynamics. 
However, as the complexity of simulated systems increases, the current YAML-based configuration system presents significant challenges that limit productivity and reliability. 
This thesis addresses these limitations by developing a type-safe Domain-Specific Language (DSL) in Kotlin for Alchemist configuration. 
The DSL leverages Kotlin's type system and IDE support to provide compile-time error detection, autocomplete capabilities, and improved developer experience, while maintaining semantic equivalence with the existing YAML configuration system.

The remainder of this thesis is organized as follows. 
Chapter~\ref{chap:background} provides background information on Alchemist, including its application domains, use cases, and the motivation that led to this work. 
Chapter~\ref{chap:analysis} analyzes the requirements and constraints for developing the DSL, examining the limitations of the current YAML approach and establishing the domain model. 
Chapter~\ref{chap:design} presents the design of the DSL, including its architecture and key design decisions. 
Chapter~\ref{chap:implementation} describes the implementation details and challenges encountered during development. 
Chapter~\ref{chap:evaluation} evaluates the DSL through case studies and comparison with YAML configurations. 
Finally, Chapter~\ref{chap:conclusion} summarizes the contributions and discusses future work.
