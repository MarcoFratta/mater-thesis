%----------------------------------------------------------------------------------------
\chapter{Introduction}
\label{chap:introduction}
%----------------------------------------------------------------------------------------

Modern software systems, from cloud infrastructure orchestrators to scientific simulation frameworks, 
increasingly rely on declarative configuration to specify their structure, behavior, and parameters.
This separation of configuration from implementation enables a single codebase to serve diverse use cases 
through different configuration artifacts, facilitating reproducibility and version control.
As systems grow in complexity, however, the configuration mechanisms themselves become a critical bottleneck.

The software industry has largely converged on a narrow set of untyped data serialization formats, 
Extensible Markup Language (XML)\footnote{\url{https://www.w3schools.com/xml/xml_whatis.asp}},
JavaScript Object Notation (JSON)\footnote{\url{https://www.json.org/json-en.html}}, 
and YAML Ain't Markup Language (YAML)\footnote{\url{https://yaml.org/}}, as the de facto standard for expressing configuration artifacts.
These formats were initially designed to offer human readability and widespread tool support, 
but their fundamental design as untyped data structures introduces severe limitations that become increasingly
problematic as configuration complexity grows.
Type information exists only implicitly through runtime validation, meaning configuration errors surface only when the system attempts
 to load or execute the configuration, potentially after significant development effort has been invested.
The absence of compile-time verification means that typos, 
mismatched parameter types, or structural inconsistencies remain undetected until runtime,
 leading to opaque error messages and time-consuming debugging sessions.

Beyond the lack of type safety, these formats provide limited mechanisms for code reuse, abstraction, and modularity.
While YAML supports anchors and aliases for within-file reuse, and XML provides standardized inclusion mechanisms, 
none of these formats offer the modularity and type-checked composition that programming languages provide.
Complex configurations often span hundreds of lines, with semantic relationships obscured by syntactic boilerplate, 
making them difficult to understand, maintain, and extend.

Domain-Specific Languages (DSLs)~\cite{fowler2010domain} have emerged as a promising approach to address these limitations.
By providing syntax and abstractions that align directly with domain concepts, DSLs enable users to express configurations 
in terms of \textit{what} to compute rather than \textit{how} to compute it, bridging the semantic gap between 
the problem domain and the implementation.

This thesis explores the design and implementation of type-safe internal DSLs for configuring complex systems, 
addressing the fundamental challenge of reconciling the static nature of type systems with the
 dynamic requirements of extensible frameworks.
The work demonstrates how modern language features, can be used to create DSLs that are both 
type-safe and highly usable, without sacrificing the flexibility required by extensible systems.

The \textit{Alchemist simulation framework}\cite{alchemistold} serves as a concrete use case for this thesis.
Alchemist is a general-purpose simulation framework designed for modeling spatially-distributed systems, 
enabling researchers to model diverse phenomena ranging from biochemical networks to distributed computing systems, 
crowd dynamics, and swarm robotics.
Currently, Alchemist simulations are configured via YAML files, following a YAML specification that allows users to have full control 
over the simulation configuration.
However, as simulation complexity grows, the limitations of YAML-based configuration become increasingly problematic, 
adversely affecting the usability and maintainability of configuration files.
This thesis presents a \textit{Kotlin-based}\footnote{\url{https://kotlinlang.org/}} DSL that aims to replace YAML 
as the primary configuration format for Alchemist, 
demonstrating how type-safe configuration languages can improve developer experience, reduce errors, and enhance maintainability 
while preserving the semantic equivalence and extensibility of the original system.

The first part of this thesis is \Cref{chap:background-motivation}, which provides background
 information and details the context and motivation behind this thesis work, 
establishing the foundational concepts of Domain-Specific Languages and the challenges of configuration management for complex systems.
\Cref{chap:analysis} analyzes the requirements and constraints for developing the DSL, 
examining the limitations of the current YAML approach and establishing the domain model that guides the DSL design,
\Cref{chap:design} presents the design of the DSL, including its architecture and key design decisions that reconcile 
type safety with the dynamic requirements of extensible frameworks.
\Cref{chap:implementation} describes the implementation details and challenges encountered during development, 
focusing on the mechanisms that enable type-safe configuration while maintaining flexibility.
\Cref{chap:evaluation} presents the evaluation of the DSL, detailing the methods and results of the assessments performed 
to verify correctness, equivalence with the legacy system, and improvements in developer experience.
Finally, \Cref{chap:conclusion} summarizes the contributions of this work and discusses potential future work.
