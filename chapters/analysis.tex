%----------------------------------------------------------------------------------------
\chapter{Analysis}
\label{chap:analysis}
%----------------------------------------------------------------------------------------

This chapter analyzes the problem of developing a \textbf{type-safe}
Domain-Specific Language for Alchemist configuration. 
The analysis establishes the foundation for the design 
and implementation phases by identifying
requirements, constraints, and the domain model underlying the simulation configuration.


The chapter \textbf{begins} by defining high-level goals for the DSL.
It then examines constraints that the solution must satisfy, 
including backward compatibility with the existing configuration system and 
limitations imposed by the \textbf{Kotlin} language.
Functional requirements specify what the DSL must accomplish,
particularly equivalence with YAML semantics and integration with Alchemist's existing
loading system. 
Non-functional requirements address usability, \textbf{extensibility},
and performance concerns.

The analysis then investigates the limitations of the current YAML-based approach and examines 
how the existing type system constrains the solution. 
Finally, the chapter models the simulation configuration domain, 
identifying key concepts and their relationships that guide the DSL design.

\section{High-level Goals}

The \textbf{primary goal} is to develop a type-safe 
Domain-Specific Language in Kotlin that replaces YAML as the main configuration format for Alchemist simulations. 
The DSL should eliminate the type safety and tooling limitations of YAML 
while maintaining \textbf{semantic equivalence} with existing YAML configurations. 
The adoption of a \textit{Programming Language} for the simulation configuration should improve the overall developer experience and productivity:
the DSL can provide compile-time type and syntax checking, enabling early detection of configuration errors before simulation execution.
Moreover, the \textit{Integrated Development Environment} (IDE) support, including autocomplete and inline documentation, 
should reduce the learning curve and improve productivity when creating simulation configurations. 
The user can leverage the IDE to focus on the simulation configuration and less on the language syntax.

The second goal concerns compatibility and integration. 
The DSL must integrate seamlessly with Alchemist's existing loading infrastructure, 
producing identical simulation models to those generated from YAML. 
This ensures that simulations configured via the DSL behave identically to their YAML counterparts,
 preserving reproducibility and allowing gradual migration from YAML to the DSL.

The \textbf{third goal} focuses on expressiveness and maintainability. 
The DSL should provide a natural, domain-appropriate syntax that makes simulation configurations more readable and maintainable 
than their YAML equivalents. However, the DSL syntax should be as close as possible to the YAML syntax, 
allowing the user to leverage their existing knowledge of YAML to configure simulations. Moreover, the DSL, 
should be open to extensions and should be able to support new features or new incarnations as they are developed, without 
the need to modify the DSL itself.


Next sections will analyze, in more detail, the constraints and functional requirements that the DSL must satisfy.
\section{Constraints}

Constraints define \textbf{limitations} and \textbf{conditions} that the DSL solution must satisfy, 
restricting the design space and implementation choices. 
These constraints arise from the need to integrate with existing systems and comply with language specifications.


Two primary constraint categories affect the DSL design:
\begin{itemize}
    \item \textbf{Backward compatibility}: ensures that the DSL integrates with Alchemist's 
    existing configuration system without breaking existing functionality. 
    \item \textbf{Language constraints}: the Kotlin language syntax and the available tooling capabilities 
    limit the implementation choices. 
\end{itemize}

These two constraints will be analyzed in more detail in the next sections.
\subsection{Backward Compatibility}
\subsection{Kotlin Language Constraints}
\section{Functional Requirements}
\subsection{YAML Equivalence}
\subsection{Loading System Requirements}
\subsection{Type Safety}
\section{Non-functional Requirements}
\subsection{DSL API Usability}
\subsection{Extensibility}
\subsection{Performance}
\section{Requirements Analysis}
\section{Problem Analysis}
\subsection{Limitations of YAML Configuration}
\subsection{Legacy Type System Limitations}
\section{Simulation Configuration Domain}

