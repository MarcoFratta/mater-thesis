%----------------------------------------------------------------------------------------
\chapter{Conclusion and Future Work}
\label{chap:conclusion}
%----------------------------------------------------------------------------------------

This thesis has presented the design and implementation of a type-safe Domain-Specific Language 
in Kotlin for configuring Alchemist simulations, 
addressing the fundamental limitations of the existing YAML-based configuration system.
 The proposed solution successfully reconciles the static type safety of the host language with the dynamic requirements 
 of an extensible simulation framework, demonstrating that declarative configuration can
  coexist with compile-time verification and a rich tooling support.


  \paragraph{Future Work}

  While the current implementation successfully addresses the core requirements, several directions for future enhancement
  present themselves.
  A natural extension would be the development of conversion tools that enable seamless migration between YAML and DSL formats. 
  A \textbf{YAML-to-DSL converter} could automatically translate existing YAML configurations into equivalent DSL scripts,
  facilitating the migration of legacy configurations.
  
  Currently, the DSL handles node programs as generic string values, 
  requiring users to write reactions using the native syntax of each 
  incarnation (e.g., \texttt{"\{token\} --> \{firing\}"}).
  While this approach maintains compatibility with the existing Alchemist architecture, 
  it sacrifices the type safety and IDE support that are core benefits of the DSL.
  
  Future work could extend the DSL to support \textbf{custom} DSLs for each incarnation, allowing users to define 
  programs using incarnation-specific syntax that is fully integrated with Kotlin's type system. 
  For example, instead of string-based 
  reaction definitions, the SAPERE incarnation could provide a DSL that allows users to define eco-laws using type-safe constructs:
  
  \begin{minted}[fontsize=\footnotesize, linenos]{kotlin}
  sapere {
      deployments {
          programs {
              ecoLaw {
                  consume("token")
                  produce("firing")
              }
          }
      }
  }
  \end{minted}
  
  This can also be extended to support custom DSLs for each incarnation, creating an adapter layer that translates the custom 
  incarnation-specific DSL into the generic DSL.
  
  


\paragraph{Conclusion}

The primary contribution of this work is a complete DSL implementation that is an alternative to YAML for configuring Alchemist simulations
while maintaining full semantic equivalence with the legacy system. 
The DSL leverages Kotlin's advanced language features, such as lambdas with receivers,
delegated properties, 
and context parameters, to create a declarative syntax that closely mirrors the domain model of Alchemist simulations.

A key technical achievement is the resolution of the challenge of allowing the usage of legacy Alchemist classes 
while maintaining usability and type safety.
Through the use of Kotlin Symbol Processing (KSP), the system automatically generates type-safe helper 
functions that bridge the gap between the simplified DSL API and the complex constructors of Alchemist components.
This generative approach ensures that the DSL remains extensible: 
any new Alchemist component can be integrated into the DSL simply by annotating it with \texttt{@AlchemistKotlinDSL},
without requiring modifications to the DSL core.

The implementation successfully addresses the batch execution requirement through a deferred 
evaluation model combined with thread-local variable resolution. 
This design allows a single DSL script to generate multiple independent simulation instances with varying parameters, 
while maintaining thread safety in concurrent execution scenarios.



The following shows a comparison between a YAML configuration and the equivalent DSL configuration:

\inputminted[
fontsize=\scriptsize,
breaklines,
linenos
]{yaml}{listings/15-variables.yml}
\captionof{lstlisting}
{Example of a YAML configuration with variables}
\label{lst:yaml-variables}
\newpage

\inputminted[
fontsize=\scriptsize,
breaklines,
linenos
]{kotlin}{listings/15-variables.alchemist.kts}
\captionof{lstlisting}
{Equivalent DSL configuration of \protect\Cref{lst:yaml-variables}}



The successful implementation of the Alchemist DSL demonstrates that type-safe configuration languages are not only 
feasible but also provide significant advantages over traditional untyped formats. 
The work shows that modern language features can be leveraged to create DSLs that are both highly usable and fully extensible,
 without sacrificing the flexibility required by complex, extensible frameworks.

The generative approach using KSP represents a scalable solution to the challenge of maintaining type safety in the
presence of dynamic class loading.
This technique could be applied to other domains where similar challenges exist between static verification and runtime flexibility.

Furthermore, the DSL's support for modularity and code reuse addresses a fundamental limitation of 
data-serialization-based configuration formats, enabling users to build libraries of reusable simulation components
 and share them across projects. This capability transforms simulation configuration from writing monolithic files
  to composing simulations from multiple building blocks.







